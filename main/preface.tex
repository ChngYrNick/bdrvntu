%!TEX root = ../bdr.tex
\chapter*{Вступ}

Сьогодні галузь розробки програмного забезпечення (ПЗ) вважається найбільшою галуззю світової економіки. За деякими оцінками на 2015 рік кількість розробників ПЗ у світі сягнула 19 мільйонів осіб, а до 2019 року прогнозується зростання їх кількості до 25 мільйонів. В Україні за попередніми оцінками на 2016 рік було налічено 99940 осіб зайнятих в IT галузі, і до цієї кількості входять не тільки програмісти, а й інженери з якості ПЗ, менеджери проектів, тощо. Інша ж частина людства безпосередньо залежить від їх успішної діяльності.

За час становлення галузі, завдання, їх складність, форми подання даних та методи їх обробки сильно змінились, але до сьогодні розробка якісного програмного забезпечення не стала нормою. Очевидно, що в галузі забезпечення якості ПЗ – криза. Великі проекти виконуються із запізненням або перевищенням кошторису витрат, розроблені програми не відповідають функціональним вимогам, ПЗ має низьку продуктивність, а якість програм не влаштовує користувачів.

Останнім часом група The Standish Group кожні два роки публікує звіт «The Chaos Report» з аналізом результатів впровадження програмних проектів. Результати аналізу показують, що у 2016 році лише 16,2 \% проектів завершились з дотриманням графіку та вклалися в кошторис витрат. Беручи до уваги той факт, що США щорічно витрачає більше ніж 250 млрд. дол. на IT галузь в цілому, на успішні проекти витрачається лише 40,5 млрд. дол. Інші кошти витрачаються на проекти, які ніколи не будуть впроваджені і не принесуть користі.

Наразі розроблено чимало засобів, методів, стандартів та технологій, які покликані забезпечити якість програмного забезпечення, але якість ПЗ, як і раніше, залежить в основному від знань та досвіду розробників. Проекти не рідко зазнають невдач через невірне формулювання вимог, невдале проектування або неефективне планування, неправильне розуміння або недостатній аналіз специфікації, і, як наслідок, помилки на ранніх етапах життєвого циклу програмного забезпечення. Тому, виявляється, що кращий спосіб підвищення якості ПЗ є мінімізація часу, який витрачається на виправлення коду, в наслідок зміни вимог, зміни проекту чи виконання відлагоджування. Це підтверджується реальними даними досліджень проведених в лабораторіях NASA, які показали, що підвищена увага до конт-ролю якості дозволяла знизити рівень помилок, але не підвищувала загальні витрати на розробку.

Проте процес розробки програмного забезпечення, як і процес оцінювання якості ПЗ залишаються такими, що вимагають системного підходу, що має наукове підґрунтя. Тому пошук шляхів покращення системи управління якістю програмного забезпечення залишається актуальною задачею.
Мета і задачі дослідження. Метою роботи є пошук шляхів вдосконалення системи управління якості програмного забезпечення в рамках існуючих методик та підходів до оцінки якості.

Для досягнення мети сформульовано наступні задачі:
\begin{itemize}
\item	провести огляд існуючих систем якості;
\item	проаналізувати методики та підходи до оцінки якості програмного забезпечення;
\item	визначити найбільш ефективні, визначити недоліки, запропонувати шляхи покращення;
\item	підтвердити справедливість зроблених припущень, сформулювати задачі подальшого дослідження.
\end{itemize}

\textit{Об’єктом} роботи є система управління якості програмного забезпечення.

\textit{Предметом} роботи є процес вдосконалення існуючої системи управління якості програмного забезпечення, підвищення ефективності методів та підходів до визначення якості ПЗ.

\textit{Практичне значення} одержаних в роботі результатів полягає в тому, що:
\begin{enumerate}
\item	в результаті аналізу існуючих методів і підходів визначені найбільш слабкі місця в сучасній системі управління якістю програмного забезпечення;
\item	запропонована множина метрик якості та діапазони їх зміни для контролю якості ПЗ на ранніх етапах життєвого циклу.
\end{enumerate}
